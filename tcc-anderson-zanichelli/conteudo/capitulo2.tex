%---------- Segundo Capitulo ----------
\chapter{Visão geral da área}
\section{Sistemas operacionais}
Atualmente existem muitas empresas colocando dispositivos móveis no mercado, cada uma delas opta por um dos sistemas operacionais diponíveis.
Existem sistemas operacionais livres e outros de código fechado, alguns são utilizados apenas em dispositivos da própria fabricante, como no caso do IOS da Apple que é utilizado apenas nos IPhones.

\subsection{Android}
O Android foi desenvolvido pela Google Inc. É um \sigla{SO}{Sistema operacional} construído sobre um kernel Linux e é o SO mais utilizado no mundo nos dispositivos móveis. Ele é de uso geral e além dos dispositivos móveis pode ser utilizado em tablets, TVs, smartwatches e até desktops. Por ser um SO Livre e de Código Aberto muitas empresas o escolhem e o modificam para ser o sistema de seus produtos.

\begin{citacao}
Android é um sistema de software de código aberto criado para uma grande variedade de dispositivos. Os objetivos primordiais do Android são a criação de uma plataforma de software aberto disponível para operadoras, OEMs e desenvolvedores para que tornem as suas ideias inovadoras em realidade e que melhorem as experiências dos usuários. (About Android)\footnote[1]{Disponível em: <https://source.android.com/source/index.html> Acesso em: Dezembro/2015}
\end{citacao}

\subsection{iOS}
O \sigla{iOS}{iPhone OS} é um sistema derivado do Darwin core OS é o SO utilizado nos dispositivos móveis da Apple Inc., o iPhone. O iOS tem a segunda maior base instalada em dispositivos moveis. É um sistema de código proprietário e de código fechado.

\begin{citacao}
Como a Apple cria tanto o hardware e o sistema operacional para iPad, iPhone e iPod touch, tudo é projetado para trabalhar em conjunto. Assim, os aplicativos tiram o máximo proveito dos recursos de hardware, como o processador dual-core, aceleração gráfica, as antenas wireless, e muito mais. (About iOS)\footnote[2]{Disponível em: <http://www.apple.com/ios/what-is/> Acesso em: Dezembro/2015}
\end{citacao}

\subsection{Windows Phone}
O Windows Phone é o SO da Microsoft para dispositivos móveis. É o terceiro SO mais utilizado em dispositivos móveis. É de código proprietário e de código fechado.

\begin{table}[!htb]
	\centering
	\caption[Sistemas operacionais mais utilizados]{Vendas no mundo de Smartphones para usuários finais por sistema operacional em 2015 (Milhares de Unidades)}
	\label{tab:OS}
	\begin{tabular}{c|c|c|c|c}
		\hline \SPACE
		\textbf{OS} & \textbf{2015 (Un)}  & \textbf{2015 Market Share(\%)}  & \textbf{2014 (Un)}  & \textbf{2014 Market Share(\%)} \\ \hline \SPACE
		Android & 271.010 & 82,2 & 243.484 & 83,8\\ \hline \SPACE
		iOS & 48.086 & 14,6 & 35.345 & 12,2\\ \hline \SPACE
		Windows Phone & 8.198 & 2,5 & 8.095 & 2,8\\ \hline \SPACE
		BlackBerry & 1.153 & 0,3 & 2.044 & 0,7\\ \hline \SPACE
		Others & 1.229 & 0,4 & 1.416,8 & 0,5\\ \hline \SPACE
		Total & 329.676,4 & 100,0 & 290.384,4 & 100,0\\
		\hline
	\end{tabular}
\end{table}\vspace{-1cm}
\begin{center}\small{Fonte: Gartner, Agosto 2015\footnote[3]{Disponível em: <http://www.gartner.com/newsroom/id/3115517> Acesso em: Dezembro/2015}}
\end{center}