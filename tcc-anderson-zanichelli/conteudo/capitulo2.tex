%---------- Segundo Capitulo ----------
\chapter{Visão geral da área}

\section{Conectividade}
Os dispositivos móveis encontrados no mercado possibilitam que o usuário possam conectá-lo à Internet de alguma forma, sempre utilizando alguma tecnologia de comunicação sem fio, sendo que as mais comuns são as redes locais sem fio \sigla{WLAN}{Wireless Local Area Network} e as redes de telefonia móvel (Celular).
\subsection{WLAN}
Ultimamente o uso da tecnologia de conexão à rede sem fios espalhou-se por todos os lugares, em casa, na universidade, no trabalho, nos aeroportos, hoteis, até em alguns restaurantes.
As redes \textit{Wireless} são conhecidas como redes Wi-Fi, elas abrangem as tecnologias 802.11 do padrão \sigla{IEEE}{Institute of Electrical and Electronics Engineers}.
\begin{citacao}
Wi-Fi é o nome associado à família do padrão IEEE 802.11\footnote{http://www.ieee802.org/11/index.shtml}. Assim como o
padrão 802.15, esse padrão opera em faixas de freqüências que não necessitam de licença para instalação e operação. Suas faixas de freqüência são 2,4GHz, 3,6GHz e 5GHz. As aplicações do padrão 802.11 diferem do padrão 802.15 quanto à utilização da rede e a mobilidade do usuário. Wi-Fi oferece alta potência de transmissão e cobre distâncias maiores, oferecendo redes sem fio locais (WLAN). (VANNI, 2009)
\end{citacao}
Uma das maiores vantagens das redes Wi-Fi é a sua compatibilidade com praticamente todos os sistemas operacionais, incluíndo dipositivos como impressoras e video-games.

As redes Wi-Fi fazem o uso das ondas de rádio para transmitir informações e os dispositivos devem ter um adaptador que traduzem os dados em sinais de rádio. Esses sinais são transmitidos através de uma antena para um decodificador conhecido como roteador, que decodifica os sinais e envia os dados para a Internet através de uma conexão com fios à rede \sigla{LAN}{Local Area Network}.

Como as redes Wireless funcionam recebendo e emitindo sinais, os dados recebidos da Internet são codificados em sinais de rádio pelo roteador e recebidos pelo adaptador do dispositivo conectado.
Os pontos de acesso às redes Wi-Fi podem ser públicas, como por exemplo nos aeroportos e restaurantes ou fechadas, como nos locais de trabalho e nas casas.

Para bloquear esse acesso às redes Wi-Fi são utilizadas diversas tecnicas de autenticação com protocolos de criptografia como por exemplo: 
\subsection{3G e 4G}

\section{Sistemas operacionais}
Atualmente existem muitas empresas colocando dispositivos móveis no mercado, cada uma delas opta por um dos sistemas operacionais diponíveis.
Existem sistemas operacionais livres e outros de código fechado, alguns são utilizados apenas em dispositivos da própria fabricante, como no caso do IOS da Apple que é utilizado apenas nos iPhones.

\subsection{Android}
O Android foi desenvolvido pela Google Inc. É um \sigla{SO}{Sistema operacional} construído sobre um kernel Linux e é o SO mais utilizado no mundo nos dispositivos móveis. Ele é de uso geral e além dos dispositivos móveis pode ser utilizado em tablets, TVs, smartwatches e até desktops. Por ser um SO Livre e de Código Aberto muitas empresas o escolhem e o modificam para ser o sistema de seus produtos.

\begin{citacao}
Android é um sistema de software de código aberto criado para uma grande variedade de dispositivos. Os objetivos primordiais do Android são a criação de uma plataforma de software aberto disponível para operadoras, OEMs e desenvolvedores para que tornem as suas ideias inovadoras em realidade e que melhorem as experiências dos usuários. (About Android)\footnote{https://source.android.com/source/index.html}
\end{citacao}

\subsection{iOS}
O \sigla{iOS}{iPhone OS} é um sistema derivado do Darwin core OS é o SO utilizado nos dispositivos móveis da Apple Inc., o iPhone. O iOS tem a segunda maior base instalada em dispositivos moveis. É um sistema de código proprietário e de código fechado.

\begin{citacao}
Como a Apple cria tanto o hardware e o sistema operacional para iPad, iPhone e iPod touch, tudo é projetado para trabalhar em conjunto. Assim, os aplicativos tiram o máximo proveito dos recursos de hardware, como o processador dual-core, aceleração gráfica, as antenas wireless, e muito mais. (About iOS)\footnote{http://www.apple.com/ios/what-is/}
\end{citacao}

\subsection{Windows Phone}
O Windows Phone é o SO da Microsoft para dispositivos móveis. É o terceiro SO mais utilizado em dispositivos móveis. É de código proprietário e de código fechado.

\begin{table}[!htb]
	\centering
	\caption[Sistemas operacionais mais utilizados]{Vendas no mundo de Smartphones para usuários finais por sistema operacional em 2015 (Milhares de Unidades)}
	\label{tab:OS}
	\begin{tabular}{c|c|c|c|c}
		\hline \SPACE
		\textbf{OS} & \textbf{2015 (Un)}  & \textbf{2015 Market Share(\%)}  & \textbf{2014 (Un)}  & \textbf{2014 Market Share(\%)} \\ \hline \SPACE
		Android & 271.010 & 82,2 & 243.484 & 83,8\\ \hline \SPACE
		iOS & 48.086 & 14,6 & 35.345 & 12,2\\ \hline \SPACE
		Windows Phone & 8.198 & 2,5 & 8.095 & 2,8\\ \hline \SPACE
		BlackBerry & 1.153 & 0,3 & 2.044 & 0,7\\ \hline \SPACE
		Others & 1.229 & 0,4 & 1.416,8 & 0,5\\ \hline \SPACE
		Total & 329.676,4 & 100,0 & 290.384,4 & 100,0\\
		\hline
	\end{tabular}
\end{table}\vspace{-1cm}
\begin{center}\small{Fonte: Gartner, Agosto 2015\footnote{http://www.gartner.com/newsroom/id/3115517}}
\end{center}